\documentclass[journal,12pt,twocolumn]{IEEEtran}

\usepackage{setspace}
\usepackage{gensymb}
\singlespacing
\usepackage[cmex10]{amsmath}

\usepackage{amsthm}

\usepackage{mathrsfs}
\usepackage{txfonts}
\usepackage{stfloats}
\usepackage{bm}
\usepackage{cite}
\usepackage{cases}
\usepackage{subfig}

\usepackage{longtable}
\usepackage{multirow}

\usepackage{enumitem}
\usepackage{mathtools}
\usepackage{steinmetz}
\usepackage{tikz}
\usepackage{circuitikz}
\usepackage{verbatim}
\usepackage{tfrupee}
\usepackage[breaklinks=true]{hyperref}
\usepackage{graphicx}
\usepackage{tkz-euclide}

\usetikzlibrary{calc,math}
\usepackage{listings}
    \usepackage{color}                                            %%
    \usepackage{array}                                            %%
    \usepackage{longtable}                                        %%
    \usepackage{calc}                                             %%
    \usepackage{multirow}                                         %%
    \usepackage{hhline}                                           %%
    \usepackage{ifthen}                                           %%
    \usepackage{lscape}     
\usepackage{multicol}
\usepackage{chngcntr}

\DeclareMathOperator*{\Res}{Res}

\renewcommand\thesection{\arabic{section}}
\renewcommand\thesubsection{\thesection.\arabic{subsection}}
\renewcommand\thesubsubsection{\thesubsection.\arabic{subsubsection}}

\renewcommand\thesectiondis{\arabic{section}}
\renewcommand\thesubsectiondis{\thesectiondis.\arabic{subsection}}
\renewcommand\thesubsubsectiondis{\thesubsectiondis.\arabic{subsubsection}}


\hyphenation{op-tical net-works semi-conduc-tor}
\def\inputGnumericTable{}                                 %%
\makeatletter
\setlength{\@fptop}{0pt}
\makeatother
\lstset{
%language=C,
frame=single, 
breaklines=true,
columns=fullflexible
}
\begin{document}

\newcommand{\BEQA}{\begin{eqnarray}}
\newcommand{\EEQA}{\end{eqnarray}}
\newcommand{\define}{\stackrel{\triangle}{=}}
\bibliographystyle{IEEEtran}
\raggedbottom
\setlength{\parindent}{0pt}
\providecommand{\mbf}{\mathbf}
\providecommand{\pr}[1]{\ensuremath{\Pr\left(#1\right)}}
\providecommand{\qfunc}[1]{\ensuremath{Q\left(#1\right)}}
\providecommand{\sbrak}[1]{\ensuremath{{}\left[#1\right]}}
\providecommand{\lsbrak}[1]{\ensuremath{{}\left[#1\right.}}
\providecommand{\rsbrak}[1]{\ensuremath{{}\left.#1\right]}}
\providecommand{\brak}[1]{\ensuremath{\left(#1\right)}}
\providecommand{\lbrak}[1]{\ensuremath{\left(#1\right.}}
\providecommand{\rbrak}[1]{\ensuremath{\left.#1\right)}}
\providecommand{\cbrak}[1]{\ensuremath{\left\{#1\right\}}}
\providecommand{\lcbrak}[1]{\ensuremath{\left\{#1\right.}}
\providecommand{\rcbrak}[1]{\ensuremath{\left.#1\right\}}}
\theoremstyle{remark}
\newtheorem{rem}{Remark}
\newcommand{\sgn}{\mathop{\mathrm{sgn}}}
\providecommand{\abs}[1]{\vert#1\vert}
\providecommand{\res}[1]{\Res\displaylimits_{#1}} 
\providecommand{\norm}[1]{\lVert#1\rVert}
%\providecommand{\norm}[1]{\lVert#1\rVert}
\providecommand{\mtx}[1]{\mathbf{#1}}
\providecommand{\mean}[1]{E[ #1 ]}
\providecommand{\fourier}{\overset{\mathcal{F}}{ \rightleftharpoons}}
%\providecommand{\hilbert}{\overset{\mathcal{H}}{ \rightleftharpoons}}
\providecommand{\system}{\overset{\mathcal{H}}{ \longleftrightarrow}}
	%\newcommand{\solution}[2]{\textbf{Solution:}{#1}}
\newcommand{\solution}{\noindent \textbf{Solution: }}
\newcommand{\cosec}{\,\text{cosec}\,}
\providecommand{\dec}[2]{\ensuremath{\overset{#1}{\underset{#2}{\gtrless}}}}
\newcommand{\myvec}[1]{\ensuremath{\begin{pmatrix}#1\end{pmatrix}}}
\newcommand{\mydet}[1]{\ensuremath{\begin{vmatrix}#1\end{vmatrix}}}
\numberwithin{equation}{subsection}
\makeatletter
\@addtoreset{figure}{problem}
\makeatother
\let\StandardTheFigure\thefigure
\let\vec\mathbf
\renewcommand{\thefigure}{\theproblem}
\def\putbox#1#2#3{\makebox[0in][l]{\makebox[#1][l]{}\raisebox{\baselineskip}[0in][0in]{\raisebox{#2}[0in][0in]{#3}}}}
     \def\rightbox#1{\makebox[0in][r]{#1}}
     \def\centbox#1{\makebox[0in]{#1}}
     \def\topbox#1{\raisebox{-\baselineskip}[0in][0in]{#1}}
     \def\midbox#1{\raisebox{-0.5\baselineskip}[0in][0in]{#1}}
\vspace{3cm}
\title{Assignment 7}
\author{Ananthoju Pranav Sai - AI20BTECH11004}
\maketitle
\newpage
\bigskip
\renewcommand{\thefigure}{\theenumi}
\renewcommand{\thetable}{\theenumi}
Download all python codes from 
\begin{lstlisting}
    https://github.com/Ananthoju-Pranav-Sai/AI1103/tree/main/Assignment_7/Codes
\end{lstlisting}
%
and latex codes from 
%
\begin{lstlisting}
    https://github.com/Ananthoju-Pranav-Sai/AI1103/blob/main/Assignment_7/main.tex
\end{lstlisting}
\section*{UGC Dec 2018 Math set A Q 114 }
    Suppose that $X_1,X_2,X_3,...,X_{10}$ are i.i.d, N(0,1). Which of the following statements are correct ?
\begin{enumerate}[label = (\Alph*)]
    \item $\pr{X_1>X_2+X_3+...+X_{10}}=\frac{1}{2}$
    \item $\pr{X_1>X_2X_3...X_{10}}=\frac{1}{2}$
    \item $\pr{\sin{X_1}>\sin{X_2}+\sin{X_3}+...+\sin{X_{10}}}=\frac{1}{2}$
    \item $\pr{\sin{X_1}>\sin{X_2}\sin{X_3}...\sin{X_{10}}}=\frac{1}{2}$
\end{enumerate}
\section*{Solution}
Given, $X_1,X_2,X_3,...,X_{10}$ are identically independent events which follow normal distribution with mean 0 and variance 1.\\
\begin{enumerate}[label = (\Alph*)]
    \item Now let $Y$ be a random variable which is defined as follows 
\begin{align}
    Y=X_2+X_3+...+X_{10}-X_{1}
\end{align}
As all $X_i$'s follow normal distribution, $Y$ will also follow normal distribution.\\
Then $\pr{X_1>X_2+X_3+...+X_{10}}$ can be written as $\pr{Y<0}$
\begin{multline}
   E(Y)=E(X_2)+E(X_3)+E(X_4)+E(X_5)+E(X_6)\\
   +E(X_7)+E(X_8)+E(X_9)+E(X_{10})-E(X_1) 
\end{multline}
\begin{align}
    \implies E(Y)&=0\\
    \implies \mu_Y&=0
\end{align}
As all $X_i$'s are independent variance of Y can be written as
\begin{multline}
   \sigma^2_Y =  \sigma^2_{X_2}+\sigma^2_{X_3}+\sigma^2_{X_4}+\sigma^2_{X_5}+\sigma^2_{X_6}+\sigma^2_{X_7}\\
   +\sigma^2_{X_8}+\sigma^2_{X_9}+\sigma^2_{X_{10}}+\sigma^2_{X_1}
\end{multline}
\begin{align}
    \implies \sigma^2_Y=10
\end{align}
We know that for normal distribution pdf would look like 
a
So, now we have pdf of Y as
\begin{align}
    f_Y(y)=\frac{1}{\sqrt{20\pi}}e^{-\frac{y^2}{20}}
\end{align}
Now,
\begin{align}
    \pr{Y<0}&=\int_{-\infty}^{0}f_Y(y) \, dy\\
    \pr{Y<0}&=\frac{2\int_{-\infty}^{0}f_Y(y) \, dy}{2}\\
    \pr{Y<0}&=\frac{\int_{-\infty}^{\infty}f_Y(y) \, dy}{2}\\
    \implies \pr{Y<0}&=\frac{1}{2}
\end{align}
\begin{align}
    \therefore \pr{X_1>X_2+X_3+...+X_{10}}=\frac{1}{2}
\end{align}
\item We will consider only three variables first and proceed to generalise the method.
\begin{align}
    &\pr{X_1>X_2X_3}\\
    = \sum_z (&\pr{X_1>z})(\pr{X_2X_3=z})\\
    = \int_{-\infty}^{\infty}&(1-F_{X_1}(z))(\pr{X_2X_3=x})\,dz\\
    = \int_{-\infty}^{\infty}&(1-F_{X_1}(z))\brak{\sum_k(\pr{X_2=z/k}\pr{X_3=k})}\,dz\\
    = \int_{-\infty}^{\infty}&(1-F_{X_1}(z))\brak{\int_{-\infty}^{\infty}f_{X_2}(z/k)f_{X_3}(k)\,dk}\,dz\\
    = \int_{-\infty}^{\infty}&(1-F_{X_1}(z))\brak{\frac{1}{2\pi}\int_{-\infty}^{\infty} e^{-\brak{\frac{z^2+k^4}{k^2}}}\,dk}\,dz
\end{align}
\end{enumerate}

\end{document}