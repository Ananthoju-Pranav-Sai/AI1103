\documentclass[journal,12pt,twocolumn]{IEEEtran}

\usepackage{setspace}
\usepackage{gensymb}
\singlespacing
\usepackage[cmex10]{amsmath}

\usepackage{amsthm}

\usepackage{mathrsfs}
\usepackage{txfonts}
\usepackage{stfloats}
\usepackage{bm}
\usepackage{cite}
\usepackage{cases}
\usepackage{subfig}

\usepackage{longtable}
\usepackage{multirow}

\usepackage{enumitem}
\usepackage{mathtools}
\usepackage{steinmetz}
\usepackage{tikz}
\usepackage{circuitikz}
\usepackage{verbatim}
\usepackage{tfrupee}
\usepackage[breaklinks=true]{hyperref}
\usepackage{graphicx}
\usepackage{tkz-euclide}

\usetikzlibrary{calc,math}
\usepackage{listings}
    \usepackage{color}                                            %%
    \usepackage{array}                                            %%
    \usepackage{longtable}                                        %%
    \usepackage{calc}                                             %%
    \usepackage{multirow}                                         %%
    \usepackage{hhline}                                           %%
    \usepackage{ifthen}                                           %%
    \usepackage{lscape}     
\usepackage{multicol}
\usepackage{chngcntr}

\DeclareMathOperator*{\Res}{Res}

\renewcommand\thesection{\arabic{section}}
\renewcommand\thesubsection{\thesection.\arabic{subsection}}
\renewcommand\thesubsubsection{\thesubsection.\arabic{subsubsection}}

\renewcommand\thesectiondis{\arabic{section}}
\renewcommand\thesubsectiondis{\thesectiondis.\arabic{subsection}}
\renewcommand\thesubsubsectiondis{\thesubsectiondis.\arabic{subsubsection}}


\hyphenation{op-tical net-works semi-conduc-tor}
\def\inputGnumericTable{}                                 %%
\makeatletter
\setlength{\@fptop}{0pt}
\makeatother
\lstset{
%language=C,
frame=single, 
breaklines=true,
columns=fullflexible
}
\begin{document}

\newcommand{\BEQA}{\begin{eqnarray}}
\newcommand{\EEQA}{\end{eqnarray}}
\newcommand{\define}{\stackrel{\triangle}{=}}
\bibliographystyle{IEEEtran}
\raggedbottom
\setlength{\parindent}{0pt}
\providecommand{\mbf}{\mathbf}
\providecommand{\pr}[1]{\ensuremath{\Pr\left(#1\right)}}
\providecommand{\qfunc}[1]{\ensuremath{Q\left(#1\right)}}
\providecommand{\sbrak}[1]{\ensuremath{{}\left[#1\right]}}
\providecommand{\lsbrak}[1]{\ensuremath{{}\left[#1\right.}}
\providecommand{\rsbrak}[1]{\ensuremath{{}\left.#1\right]}}
\providecommand{\brak}[1]{\ensuremath{\left(#1\right)}}
\providecommand{\lbrak}[1]{\ensuremath{\left(#1\right.}}
\providecommand{\rbrak}[1]{\ensuremath{\left.#1\right)}}
\providecommand{\cbrak}[1]{\ensuremath{\left\{#1\right\}}}
\providecommand{\lcbrak}[1]{\ensuremath{\left\{#1\right.}}
\providecommand{\rcbrak}[1]{\ensuremath{\left.#1\right\}}}
\theoremstyle{remark}
\newtheorem{rem}{Remark}
\newcommand{\sgn}{\mathop{\mathrm{sgn}}}
\providecommand{\abs}[1]{\vert#1\vert}
\providecommand{\res}[1]{\Res\displaylimits_{#1}} 
\providecommand{\norm}[1]{\lVert#1\rVert}
%\providecommand{\norm}[1]{\lVert#1\rVert}
\providecommand{\mtx}[1]{\mathbf{#1}}
\providecommand{\mean}[1]{E[ #1 ]}
\providecommand{\fourier}{\overset{\mathcal{F}}{ \rightleftharpoons}}
%\providecommand{\hilbert}{\overset{\mathcal{H}}{ \rightleftharpoons}}
\providecommand{\system}{\overset{\mathcal{H}}{ \longleftrightarrow}}
	%\newcommand{\solution}[2]{\textbf{Solution:}{#1}}
\newcommand{\solution}{\noindent \textbf{Solution: }}
\newcommand{\cosec}{\,\text{cosec}\,}
\providecommand{\dec}[2]{\ensuremath{\overset{#1}{\underset{#2}{\gtrless}}}}
\newcommand{\myvec}[1]{\ensuremath{\begin{pmatrix}#1\end{pmatrix}}}
\newcommand{\mydet}[1]{\ensuremath{\begin{vmatrix}#1\end{vmatrix}}}
\numberwithin{equation}{subsection}
\makeatletter
\@addtoreset{figure}{problem}
\makeatother
\let\StandardTheFigure\thefigure
\let\vec\mathbf
\renewcommand{\thefigure}{\theproblem}
\def\putbox#1#2#3{\makebox[0in][l]{\makebox[#1][l]{}\raisebox{\baselineskip}[0in][0in]{\raisebox{#2}[0in][0in]{#3}}}}
     \def\rightbox#1{\makebox[0in][r]{#1}}
     \def\centbox#1{\makebox[0in]{#1}}
     \def\topbox#1{\raisebox{-\baselineskip}[0in][0in]{#1}}
     \def\midbox#1{\raisebox{-0.5\baselineskip}[0in][0in]{#1}}
\vspace{3cm}
\title{Assignment 8}
\author{Ananthoju Pranav Sai - AI20BTECH11004}
\maketitle
\newpage
\bigskip
\renewcommand{\thefigure}{\theenumi}
\renewcommand{\thetable}{\theenumi}
Download the latex code from 
%
\begin{lstlisting}
https://github.com/Ananthoju-Pranav-Sai/AI1103/blob/main/Assignment_8/main.tex
\end{lstlisting}
\section*{UGC June 2017 Math set A Q 57 }
Suppose $(X_1,X_2)$ follows a bivariate  normal distribution with
\begin{align}
    &E(X_1)=E(X_2)=0\\
    &V(X_1)=V(X_2)=2\\
    &Cov(X_1,X_2)=-1
\end{align}
If $\Phi(x)$=$\frac{1}{ \sqrt{2\pi}}\int_{-\infty}^{x}e^{-\frac{{y}^2}{2}}\, dy$, then $\pr{X_1-X_2>6}$ = ?

\begin{enumerate}
    \item $\Phi(-1)$
    \item $\Phi(-3)$
    \item $\Phi(\sqrt{6})$
    \item $\Phi(-\sqrt{6})$
\end{enumerate}
\section*{Solution}
Given, $(X_1,X_2)$ follows a bivariate normal distribution with
\begin{align}
    \mu_1=\mu_2&=0\\
    \sigma^2_1=\sigma^2_2&=2\\
    V_{12}&=-1\\
    \rho=\frac{V_{12}}{\sigma_1\sigma_2}&=\frac{-1}{2}
\end{align}
where $\rho$ is correlation of $x_1$ and $x_2$\\
We define mean matrix $\vec{\mu}$
\begin{align}
    \vec{\mu}&=\myvec{\mu_1\\
             \mu_2}
             =\myvec{0\\
             0}
             \label{a}
\end{align}
 and covariance matrix \vec{$\Sigma$} as follows
\begin{align}
    \vec{\Sigma}&=\myvec{\sigma_1^2&\rho\sigma_1\sigma_2\\
                    \rho\sigma_1\sigma_2&\sigma_2^2}
                = \myvec{2&-1\\
                        -1&2}
                \label{b}
\end{align}
The characteristic function of bivariate is given by
\begin{align}
    \phi_{X_1 X_2}(t_1,t_2)=\mathbb{E}[e^{i(t_1X_1+t_2X_2)}]
    \label{c}
\end{align}
which can also be written as
\begin{align}
    \phi_{X_1 X_2}(t_1,t_2)=\exp({i\vec{u}^{\top}\vec{\mu}-\frac{1}{2}\vec{u}^{\top}\vec{\Sigma}\vec{u}})
    \label{d}
\end{align}
where
\begin{align}
    \vec{u}&=\myvec{t_1\\
                   t_2}
\end{align}
by using \eqref{a} and \eqref{b} in \eqref{c} we get
\begin{align}
   \phi_{X_1 X_2}(t_1,t_2)&=\exp\brak{0-\frac{1}{2}\myvec{t_1&t_2}\myvec{2&-1\\
                                                         -1&2}\myvec{t_1\\
                                                                    t_2}}\\
    \implies \phi_{X_1 X_2}(t_1,t_2)&=\exp\brak{-\frac{1}{2}\myvec{t_1&t_2}\myvec{2t_1-t_2\\
                                                         2t_2-t_1}}\\
    \implies \phi_{X_1 X_2}(t_1,t_2)&=e^{-\brak{t_1^2-t_1t_2+t_2^2}}
    \label{e}
\end{align}
Now consider a random variable Z defined as follows
\begin{align}
    Z=X_2-X_1
\end{align}
Then $\pr{X_1-X_2>6}$ can be written as $\pr{Z<-6}$\\
characteristic function of Z can be written as
\begin{align}
    \phi_Z(t)&=\mathbb{E}[e^{itZ}]\\
    \implies \phi_Z(t)&=\mathbb{E}[e^{it(X_2-X_1)}]
    \label{f}
\end{align}
clearly form \eqref{c}, \eqref{f} can be written as
\begin{align}
    \phi_Z(t)&=\phi_{X_1 X_2}(-t,t)
\end{align}
from \eqref{e}
\begin{align}
    \phi_Z(t)&=e^{-\brak{t^2+t^2+t^2}}\\
    \implies \phi_Z(t)&=e^{-3t^2}
\end{align}
Now we can find pdf of Z by using a variation of Gil-Peleaz inversion formula
\begin{align}
    f_Z(z)&=\frac{1}{2\pi}\int_{-\infty}^{\infty}e^{-itz}\phi_Z(t)\,dt\\
    \implies f_Z(z)&=\frac{1}{2\pi}\int_{-\infty}^{\infty}e^{\brak{-3t^2-itz}}\,dt
    \label{g}
\end{align}
Using the result of Gaussian integral
\begin{align}
    \int_{-\infty}^{\infty}e^{-ax^2+bx}\,dx &= e^{b^2/4a}\sqrt{\frac{\pi}{a}}
\end{align}
in \eqref{g}
\begin{align}
    \implies f_Z(z) &= \frac{1}{2\pi}e^{\frac{(-iz)^2}{12}}\sqrt{\frac{\pi}{3}}\\
    \implies f_Z(z) &= \frac{1}{\sqrt{12\pi}}e^{-\frac{{z}^2}{12}}
\end{align}
Now for $\pr{Z<-6}$
\begin{align}
    \pr{Z<-6}&=\int_{-\infty}^{-6}f_Z(z)\,dz\\
             &=\int_{-\infty}^{-6}\frac{1}{ \sqrt{12\pi}}e^{-\frac{{z}^2}{12}}\,dz\\
             &=\frac{1}{ \sqrt{2\pi}}\int_{-\infty}^{-6}\frac{1}{ \sqrt{6}}e^{-\frac{{z}^2}{12}}\,dz
             \label{h}
\end{align}
let $y=\frac{z}{\sqrt{6}}$ then \eqref{h} can be written as
\begin{align}
    \pr{Z<-6}&=\frac{1}{ \sqrt{2\pi}}\int_{-\infty}^{-\sqrt{6}}e^{-\frac{{y}^2}{2}}\,dy\\
    \implies \pr{Z<-6}&=\Phi(-\sqrt{6})\\
    \therefore \pr{X_1-X_2>6}&=\Phi(-\sqrt{6})
\end{align}
\end{document}